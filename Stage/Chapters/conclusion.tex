\chapter{Conclusion} % Main chapter title

\label{Chapter4} % Change X to a consecutive number; for referencing this chapter elsewhere, use \ref{ChapterX}

\lhead{Chapitre 4. \emph{Conclusion}} % Change X to a consecutive number; this is for the header on each page - perhaps a shortened title
Tout au long de ce rapport, j'ai essayé de vous présenter le travail que j'ai pu réaliser en entreprise qui consiste à mettre en place un outil de travail collaboratif.
On peut dire que le travail a été abouti même s'il y a des fonctionnalités qui n'ont pas été terminées à cause des contraintes que j'ai définies au niveau de la section~\ref{nonrealise}.
Ce projet m'a été beaucoup bénéfique sur le plan technique, car ça m'a permis de pouvoir mettre en pratique certaines connaissances que j'ai apprises en classe et aussi d'apprendre un nouveau CMS \og Drupal \fg{} qui était inconnu pour moi avant mon stage.
Par rapport à l'outil il y a certains points qu'on pourrait améliorer notamment installer un outil d'accélérateur de serveur comme vanish afin que le site soit encore plus rapide. On pourra également essayer de développer à côté un module supportant l'authentification GARDIAN SESAME plutôt que de modifier directement dans le noyau de la connexion drupal.